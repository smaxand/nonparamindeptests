


\documentclass[11pt]{article}
\usepackage{a4wide, xspace}
%\usepackage[ngerman, english]{babel}
\usepackage{amsmath,pifont,amsfonts,amssymb,latexsym}
\usepackage{theorem}
\usepackage{fancyhdr,enumerate}
\usepackage{nicefrac}
\usepackage{natbib}
\usepackage[hidelinks]{hyperref}
\usepackage[pdftex]{graphicx}
\usepackage{pdflscape}
\usepackage{bm}
\usepackage{booktabs}
\usepackage[font=footnotesize]{caption}
\newcommand\numberthis{\addtocounter{equation}{1}\tag{\theequation}}
%%%%%%%%%%%%%%%%%%%%%%%%%%%%%%%%%%%%%%%%%%%%%%%%%%%%%%%
%NEWCOMMANDS
%
%\renewcommand{\baselinestretch}{1.5}




\begin{document}
	\section*{Nonparametric Tests for Independence in R}
	From \cite{HerMax18}, tests for different types of dependence structures:
	\begin{itemize}
		\item[1.] {\bf Bivariate dependence:} Testing for dependence between two random variables $x_1$ and $x_2$. The corresponding null hypothesis is $H_0: F_{x_1,x_2}(x_1,x_2)=F_{x_1}(x_1)F_{x_2}(x_2)$ with joint distribution function $F_{x_1,x_2}$ and marginals $F_{x_1},$ $F_{x_2}.$ 
		\item[2.]{\bf Groupwise dependence:} Analyzing two sets of variables can be thought of as a generalization of bivariate dependence tests where two disjoint subsets of $\{x_1,\ldots,x_p\}$ are subjected to testing, i.e., $\bm{x}_1\in\mathbb{R}^{p_1}$ and $\bm{x}_2\in\mathbb{R}^{p_2}$ such that $p_1+p_2=p$. The corresponding null hypothesis is $H_0: F_{\bm{x}_1,\bm{x}_2}(\bm{x}_1,\bm{x}_2)=F_{\bm{x}_1}(\bm{x}_1)F_{\bm{x}_2}(\bm{x}_2)$ for multivariate distribution functions $F_{\bm{x}_1,\bm{x}_2},\, F_{\bm{x}_1}$ and $F_{\bm{x}_2}$. Furthermore, some tests allow to diagnose the dependence between more than two disjoint subsets, where $p_1+\ldots+p_c=p$ and $c>2.$ 
		\item[3.]{\bf Mutual dependence:} To test for overall independence within a set of random variables $\{x_1,\ldots,x_p\}$ the null hypothesis is formulated as $H_0: F_{x_1,\ldots,x_p}(x_1,\ldots,x_p)=F_{x_1}(x_1)\cdots F_{x_p}(x_p).$ The tests exploit the fact that mutual independence is equivalent to independence within all subsets of $\{x_1,\ldots,x_p\}$. This hypothesis is equivalent to stating groupwise independence and choosing subsets of size $p_1=p_2=\ldots=p_c=1$. 
	\end{itemize}
	
	
	
\begin{table}[ht]
 \caption{\textsf{R} packages and functions corresponding to the procedures described in Section 3 of \cite{HerMax18}. }
\label{tab:Rpack}\small{\centering
  \begin{tabular}{lllll}\toprule
 
   &classical  & spatial rank  &  empirical copula  & distance covariance \\    
		\midrule\midrule
\textsf{R} package& \textit{Hmisc} &\textit{SpatialNP} &\textit{copula} &\textit{energy}  \\
&\citep{Hmisc}&\citep{SpatialNP}&\citep{copula}&\citep{energy}\\
&&&&\textit{steadyICA} \\
&&&&\citep{steadyICA}\\
\midrule
\multicolumn{3}{l}{\textsf{R} function for distinct dependence levels}&&\\
 bivariate &\texttt{hoeffd} ($d$)&\texttt{sr.indep.test} ($sr$)&\texttt{indepTest} ($B$)&\texttt{indep.test} ($dCov$)\\
 mutual &Wilks' Lambda ($L$)& Fisher comb. of $sr$&\texttt{indepTest} ($W$, $B$)&\texttt{permTest} ($dCov$)\\
 groupwise &Wilks' Lambda ($L$)&\texttt{sr.indep.test} ($sr$)&\texttt{multIndepTest} ($B$)&\texttt{indep.test} ($dCov$)\\
  \bottomrule
  \end{tabular}                     }
\end{table}


%\nocite{*}
\bibliographystyle{newapa}
\bibliography{lit_nonparamindep}


\end{document}